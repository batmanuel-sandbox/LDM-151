\section{Software Primitives}
\label{sec:software-primitives}

\subsection{Cartesian Geometry}
\label{sec:spCartesianGeometry}

\subsection{Points}
\label{sec:spCartesianPoints}

\subsection{Boxes}
\label{sec:spCartesianBoxes}

\subsection{Polygons}
\label{sec:spCartesianPolygons}

\subsection{Ellipses}
\label{sec:spCartesianEllipses}

\subsection{Spherical Geometry}
\label{sec:spSphericalGeometry}

\subsection{Points}
\label{sec:spSphericalPoints}

\subsection{Boxes}
\label{sec:spSphericalBoxes}

\subsection{Polygons}
\label{sec:spSphericalPolygons}

\subsection{Images}
\label{sec:spImages}

\subsubsection{Simple Images}
\label{sec:spImagesSimple}

\subsubsection{Masks}
\label{sec:spImagesMasks}

\subsubsection{MaskedImages}
\label{sec:spImagesMaskedImages}

\subsubsection{Exposure}
\label{sec:spImagesExposure}

\begin{description}
\item[Image] A 2-d array of calibrated, background-subtracted pixel values in counts.
\item[Mask] A boolean representation of artifacts, detections, saturation, and other image.  This may include (but is not limited to) a 2-d integer arrays with bits interpreted as different ``mask planes''; it may also include using \hyperref[sec:spFootprints]{Footprints} to describe labeled regions.
\item[Variance] A representation of the uncertainty in the image.  This includes at least a 2-d array capturing the variance in each pixel, and it may involve some other scheme to capture the variance.
\item[Background] An object describing the background model that was subtracted from the image; the original unsubtracted image can be obtained by adding an image of this model to the Exposure's image plane.  Backgrounds are more complex than merely an image or even an interpolated binned image; background estimation will proceed in several stages, and these stages (which may happen in different coordinate systems) must be combined to form the full background model.
\item[PSF] A model of the PSF; see \hyperref[sec:spPSF]{PSF}.  This includes a model for aperture corrections.
\item[WCS] The astrometric solution that related the image's pixel coordinate system to coordinates on the sky; see \hyperref[sec:spWCS]{WCS}.
\item[PhotoCalib] The photometric solution that relates the image's pixel values to magnitudes as a function of source wavelength or SED.  Some PhotoCalibs may represent global calibration and some may represent relative calibration.
\end{description}

\subsection{Multi-Type Associative Containers}
\label{sec:spAssociativeContainers}

\subsection{Tables}
\label{sec:spTables}

\subsubsection{Source}
\label{sec:spTablesSource}

\subsubsection{Object}
\label{sec:spTablesObject}

\subsubsection{Reference}
\label{sec:spTablesReference}

\subsubsection{Joins}
\label{sec:spTablesJoins}

\subsubsection{Queries}
\label{sec:spTablesQueries}

\subsection{Footprints}
\label{sec:spFootprints}

\subsubsection{PixelRegions}
\label{sec:spFootprintsPixelRegions}

\subsubsection{Functors}
\label{sec:spFootprintsFunctors}

\subsubsection{Peaks}
\label{sec:spFootprintsPeaks}

\subsubsection{FootprintSets}
\label{sec:spFootprintsSets}

\subsubsection{HeavyFootprints}
\label{sec:spFootprintsHeavy}

\subsubsection{Thresholding}
\label{sec:spFootprintsThresholding}

\subsection{Basic Statistics}
\label{sec:spStatistics}

\subsection{Chromaticity Utilities}
\label{sec:spChromaticity}

\subsubsection{Filters}
\label{sec:spChromaticityFilters}

\subsubsection{SEDs}
\label{sec:spChromaticitySEDs}

\subsubsection{Color Terms}
\label{sec:spColorTerms}

\subsection{PhotoCalib}
\label{sec:spPhotoCalib}

\subsection{Convolution Kernels}
\label{sec:spKernels}

\begin{itemize}
\item Supports spatially-varying convolution with a variety of tricks for special kernels (e.g. spatially varying linear combinations of fixed kernels, kernels separable in x and y).
\item Must support correlation as well.
\item Closely related to PSFs, but kernels are not wavelength dependent, and PSFs are.
\item Closely related to image resampling.  Can a resampling kernel be a Kernel?  Implies that output pixel grid is different from input pixel grid.
\item Should be able to compose Kernels.
\end{itemize}

\subsection{Coordinate Transformations}
\label{sec:spWCS}

\subsection{Numerical Integration}
\label{sec:spIntegration}

\subsection{Random Number Generation}
\label{sec:spRandomNumbers}

\subsection{Interpolation and Approximation of 2-D Fields}
\label{sec:spInterpApprox}

\subsection{Pixel Interpolation}
\label{sec:spPixelInterpolation}

\subsection{Common Functions and source Profiles}
\label{sec:spFunctions}

\subsection{Camera Descriptions}
\label{sec:spCameraDescriptions}

\subsection{Numerical Optimization}
\label{sec:spOptimization}

\subsection{Monte Carlo Sampling}
\label{sec:spMonte Carlo}

\subsection{Point-Spread Functions}
\label{sec:spPSF}

Includes aperture corrections.

\subsubsection{N-Way Matching}
\label{sec:spNWayMatching}
AUTHOR: MWV
\begin{itemize}
\item Match sources and associate objects from M catalogs each with $\sim$N sources.  The API should match in either (x, y) or (RA, Dec).  Positions for source detections solutions will be assumed to already be correct.  Order of individual catalogs should not matter.  Algorithm will need to be able to run on M$\sim$1,000 visits.  Such a tool will allow flexible analyses without the requirement for a larger database structure or full coadd-based object identifiction and forced photometry.  Even within the framework of a complete Level-2 DRP release, such a N-way matching capability will also be important for comparing the results of single-visit photometry with the deep coadd-based object detection and forced photometry.  A specific example use case for lightweight quality assessment is taking the processed catalogs for M=1,000 images each with N=2,000 sources and creating object associations add derived repeatability and time-variable summary statistics.  This algorithm and associated API should provide a general purpose tool useful for algorithm developers, data quality assessment, and science users.  A trivial in-memory version (using full catalogs), a streamlined in-memory version (load only the coordinates), and a larger-than-memory version will each be useful and important and will entail increasingly more significant design and performance efforts.
\end{itemize}
