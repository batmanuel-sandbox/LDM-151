\section{Data Access, the Science User Interface and Tools}
\label{sec:suit}

An integral part of the LSST Data Management (DM) system is the provision of user access to the 
LSST data products. Users will be able to access and interact with the data products hosted at 
Data Access Centers (DACs) through a layered set of graphical and programmatic interfaces that 
may be invoked both on the DAC systems and on remote computers.

The present document focuses on the scientific pipelines and algorithms necessary to produce 
the data described in the Data Products Definition Document. Details of the requirements, design, 
and implementation plan for the data access interfaces and the DACs themselves thus are set forth 
elsewhere. Nevertheless, we provide a short description of the expected user interfaces here in order 
to place their design in context with the rest of DM.

Access to the LSST data products, including the raw image data, the Level 1, Level 2, and Calibration 
catalog and image data products, the Engineering and Facilities Database, and Level 3 data products 
contributed by users, is enabled by two major software components: the Data Access layer (DAX) and 
the Science User Interface and Tools (SUIT). This software will run on a combination of the 
project-provided Data Access Centers' hardware and on users' remote devices, and will be supported 
at the DACs by a number of IT-infrastructure services such as authentication and authorization, 
resource provisioning, and the like.

\subsection{Data Access Layer} 

The Data Access Layer (DAX) will consist of both network APIs and Python APIs providing access to 
the above data products. They will also enable the storage and retrieval of Level 3 data products created 
by the community, depending on the availability of the resources needed for a product or set of products, 
supporting the sharing of such data products between individuals, within collaborations, or with the 
LSST community as a whole. DAX APIs will be provided both for image and tabular (e.g., catalog) data. 
The DAX layer will include support for a variety of widely-used VO services and protocols, as well as 
LSST-specific interfaces where appropriate. The Python APIs will enable access to LSST data products 
at the Data Access Centers from remote systems as well as to user processes running at the DACs. 
The DAX interfaces will provide access to a capability to run additional user-specific computations next 
to the data, operating on the results of large-scale database queries.

\subsection{Science User Interface and Tools} 

The Science User Interface and Tools (SUIT) will provide an interactive, exploratory analysis and visualization 
environment for the users of LSST data. It is designed to enable creativity and flexible use by the astronomical 
community, and will be highly configurable and customizable by individuals and collaborations. The SUIT will 
provide a web-based entry portal suitable for the discovery, searching, and exploration of the LSST data products 
and the interactive visualization of images, tabular data, and plots, while also providing a workspace environment 
accessible through the portal or programmatically for additional data manipulation, analysis, and the storage and 
retrieval of results. The workspace environment will enable users to utilize the resources made available in the 
DAC for additional processing and storage of data and results. The workspace environment will feature a strong 
integration of the SUIT's UI components for data searching and visualization with a Python Jupyter notebook 
environment, in which users will be able to take advantage of the full power of the LSST software stack as well as 
common community scientific and astronomical Python libraries and tools. The SUIT's components will be available 
through both JavaScript and Python APIs to facilitate the development of 
custom portals and the integration of these components into individuals' scientific workflows.

The DAX and SUIT interfaces and frameworks are being designed so that they are suitable for use in the quality 
assessment of the results from the DM algorithmic pipelines, as well as for operational data quality assessment 
during observing. This will take full advantage of their configurability, and the resulting system will provide 
pre-defined views of standard metrics and informative displays as well as an interactive exploration and 
visualization (``drill down'') capability.

All of the DAX and SUIT software will be part of the open-source code base of the LSST project, and much of this 
software will be developed to be widely useful beyond the project-provided DAC environment.

