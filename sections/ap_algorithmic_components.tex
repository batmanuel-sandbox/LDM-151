\section{Algorithmic Components: AP}

\subsection{Instrument Signature Removal}
\begin{itemize}
\item Mask defects and saturation
\item Assembly
\item Full frame corrections: Dark, Flats (includes fringing)
\item Pixel level corrections: Brighter fatter, static pixel size effects
\item Interpolation of defects and saturation
\item CR rejection
\item Generate snap difference
\item Snap combination
\end{itemize}



\subsection{PSF Deterimination and Background Determination -- single chip}
\begin{itemize}
\item Low order background estimation
\item Source detection
\item Selection of PSF candidate stars
\item PSF determination
\item Mask detected source
\item Re-estimate background and re-detect: iterate to convergence
\item Determine final PSF
\end{itemize}

\subsection{Photometric and Astrometric Calibration -- Single Visit}
\begin{itemize}
\item Source measurement (Aperture Corrections)
\item Source association
\item Calculate zeropoint
\item Remove known astrometric distortions
\item Fit remaining residual
\item Produce composed astrometric solution
\item Output information for OCS telemetry: ACTION clarify OCS interactions
\end{itemize}

\subsection{Generate Diffim Template for a Visit}
\begin{itemize}
\item Determine appropriate template to use
\item Generate template for observation
\end{itemize}

\subsection{Image Differencing}
\begin{itemize}
\item Obtain measurements of coadd sources
\item Determine relative astrometric solution
\item Warp template and measurements to science image frame
\item Obtain science image PSF
\item Correlate science image with science PSF
\item Deterimine appropriate PSF matching sources
\item Compute PSF matching model
\item Difference science and template images
\item Apply correction for correlated noise
\item Difference image source detection
\item Difference image source measurement: dipole fit, trailed source measurement
\item Measure flux on snap difference for all DIASources
\item Spuriousness calculation
\end{itemize}

\subsection{Ephemeris Calculation}
\begin{itemize}
\item Calculate positions for all solar system objects that may overlap the current exposure.
\end{itemize}

\subsection{DIASource Association}
\begin{itemize}
\item Match all DIASources to predicted Solar System object positions and DIAObject catalog positions
\item Perform forced photometry of un-associated DIAObjects.(Maybe not if we force photometer all DIAObjects?).
SSObjects will not be force photometered because
the precision of the prediction will not be good enough.  Force photometry for external DIAObjects?
\item Update associated DIAObjects with aggregate quantities: e.g. parallax, proper motion, and variability
metrics
\item New spuriousness calculation
\end{itemize}

\subsection{Forced Photometry}
\begin{itemize}
\item Compute point source photometry at a specified location on a specified image.
\end{itemize}

\subsection{Make Tracklets}
\begin{itemize}
\item Make all tracklet pairs
\item Merge multiple chained observation into single longer tracklets
\item Purge any tracklets inconsistent with the merged tracklets
\end{itemize}

\subsection{Attribution and precovery}
\begin{itemize}
\item Predict locations of known Solar System objects
\item Match tracklet observation to predicted ephimerides taking into account velocity
\item Update SSObjects
\item Possibly iterate
\end{itemize}

\subsection{Orbit Fitting}
\begin{itemize}
\item Merge unassociated tracklets into tracks.
\item Fit orbits to all tracks.
\item Purge unphysical tracks.
\item Update SSObjects
\item Possibly iterate
\end{itemize}