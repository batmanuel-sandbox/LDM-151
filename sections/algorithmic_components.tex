\section{Algorithmic Components}
\label{sec:algorithmic-components}

\subsection{Instrument Signature Removal}
AUTHOR: Merlin
\begin{itemize}
\item Mask defects and saturation
\item Assembly
\item Overscan
\item Linearity
\item Crosstalk
\item Full frame corrections: Dark, Flats (includes fringing)
\item Pixel level corrections: Brighter fatter, static pixel size effects
\item Interpolation of defects and saturation
\item CR rejection
\item Generate snap difference
\item Snap combination
\end{itemize}

\subsubsection{AP: just skip some steps?}
AUTHOR: Simon

\subsubsection{DRP: do all the steps}
AUTHOR: Merlin


\subsection{Artifact Detection}

\subsubsection{Single-Exposure Morphology}
AUTHOR: Simon
\begin{itemize}
\item Find CRs via morphology.
\item Find satellites via Hough transform.
\item Find some optical ghosts (etc?) from bright star catalog and optics predictions.
\end{itemize}

\subsubsection{Snap Subtraction}
AUTHOR: Simon
\begin{itemize}
\item All of the above, but improve by looking at both snaps.
\end{itemize}

\subsubsection{Warped Image Comparison}
AUTHOR: Jim
\begin{itemize}
\item Find more optical artifacts by looking at differences between warped images (this is run during background matching).
\item Find transient astronomical sources we don't want to include in coadds.
\end{itemize}

\subsection{Artifact Masking/Interpolation}
AUTHOR: Jim
\begin{itemize}
\item Set mask planes for all artifacts.
\item Eliminate small artifacts by interpolating them.
\end{itemize}

\subsection{Source Detection}
AUTHOR: Jim
\begin{itemize}
\item Detect above-threshold regions and peaks in direct or difference images.
\item Needs to work on preconvolved and unconvolved images.
\end{itemize}

\subsection{Deblending}
AUTHOR: Jim
\subsubsection{Single Visit Deblending}
\begin{itemize}
\item Generate HeavyFootprint deblends using only a single image.
\end{itemize}
\subsubsection{Multi-Coadd Deblending}
\begin{itemize}
\item Generate consistent HeavyFootprint deblends from coadds over multiple bands and possibly epoch ranges.
\end{itemize}

\subsection{Measurement}
AUTHOR: Jim
\begin{note}[NOTE]
Each bullet here is really a subsubsubsection, but Latex doesn't like that.  Could we bump up Sections to Chapters or Parts to get another level, or do something equivalent?
\end{note}

\begin{note}[Measurement Algorithms Table]
Matrix of measurement algorithms and the contexts in which they're run, indicating which combinations are supported
\end{note}

\subsubsection{Variants}
\begin{itemize}
\item Single Visit Measurement
\item Multi-Coadd Measurement
\item Forced Measurement
\item Difference Image Measurement
\item Multi-Epoch Measurement
\end{itemize}
\subsubsection{Algorithms}
\begin{itemize}
\item Centroids
\item Second-Moment Shapes
\item Aperture Photometry
\item Kron Apertures
\item Petrosian Apertures
\item Galaxy Models
\item Moving Star Models
\item Trailed Point Source Models
\item Dipole Models
\item Spuriousness
\end{itemize}
\subsubsection{Blended Measurement}
\begin{itemize}
\item Deblend Template Projection
\item Neighbor Noise Replacement
\item Simultaneous Fitting
\item Hybrid Models
\end{itemize}

\subsection{Background Estimation}
AUTHOR: Simon
\begin{itemize}
\item Fit or interpolate large-scale variations while masking out detections.
\item Needs to work in crowded fields.
\item Needs to work on both difference images and direct images.
\item Need to be able to compose backgrounds measured in different coordinate systems on different scales.
\item Needs to work on single CCDs for AP even if we use full FoV in DRP.
\end{itemize}

\subsection{Build Background Reference}
AUTHOR: Simon
\begin{itemize}
\item Given multiple overlapping visit images (already warped to a common coordinate system), synthesize a continuous single-epoch image that can be used as a reference for background matching.
\end{itemize}

\subsection{PSF Estimation}

\subsubsection{Single CCD PSF Estimation}
AUTHOR: Simon
\begin{itemize}
\item Fit simple empirical PSF model to stars from a single exposure.
\item No chromaticity.
\item May use external star catalog, but doesn't rely on one.
\item Used in Alert Production and BootstrapImChar in DRP.
\end{itemize}

\subsubsection{Full Visit PSF Estimation}
AUTHOR: Jim
\begin{itemize}
\item Decompose PSF into optical + atmosphere.
\item Constrain model with stars, telemetry, and wavefront data.
\item Wavelength-dependent.
\item Used in RefineImChar in DRP.
\item Must include some approach to dealing with wings of bright stars.
\end{itemize}

\subsection{Aperture Correction}
AUTHOR: Jim
\begin{itemize}
\item Measure curves of growth from bright stars.
\item Correct various flux measurements to infinite.
\item Propagate uncertainty in aperture correction to corrected fluxes; covariance is tricky.
\end{itemize}

\subsection{Astrometric Solutions}
AUTHOR: Simon
\subsubsection{Single CCD (for AP)}
\begin{itemize}
\item If this uses DRP's internal reference catalog, this does all we need. THIS IS A NEW DEPENDENCY BETWEEN DRP AND AP.
\end{itemize}
\subsubsection{Single Visit}
\begin{itemize}
\item Fit multi-component WCS to all CCDs in a single visit simultaneously after matching to reference catalog.
\end{itemize}
\subsubsection{Joint Multi-Visit}
\begin{itemize}
\item Fit multi-component WCS to all CCDs from multiple visits simultaneously after matching to reference catalog.
\end{itemize}

\subsection{Photometric Solutions}
AUTHOR: Simon (and Merlin?)
\subsubsection{Single CCD (for AP)}
\subsubsection{Single Visit}
\begin{itemize}
\item Fit zeropoint (and some small spatial variation?) to all CCDs simultaneously after matching to reference catalog.
\item Need for chromatic dependence unclear; probably driven by AP.
\end{itemize}
\subsubsection{Joint Multi-Visit}
\begin{itemize}
\item Derive SEDs for calibration stars from colors and reference catalog classifications.
\item Utilize additional information from wavelenth dependent photometric calibration built by calibration products production.
\item Fit zeropoint and possibly perturbations to all CCDs on multiple visits simultaneously after matching to reference catalog.
\end{itemize}

\subsection{Generate Diffim Template for a Visit}
AUTHOR: Simon
\begin{itemize}
\item Determine appropriate template to use
\item Generate template for observation (may include DCR correction)
\end{itemize}

\subsection{PSF Matching}
AUTHOR: Simon
\subsubsection{Image Subtraction}
\begin{itemize}
\item Match template image to science image, as in Alert Production and DRP Difference Image processing.
\item Includes identifying sources to use to determine matching kernel, fitting the kernel, and convolving by it.
\end{itemize}
\subsubsection{PSF Homogenization for Coaddition}
\begin{itemize}
\item Match science image to predetermined analytic PSF, as in PSF-matched coaddition.
\end{itemize}

\subsection{Image Warping}
AUTHOR: Jim
\subsubsection{Oversampled Images}
\begin{itemize}
\item Just use Lanczos.
\end{itemize}
\subsubsection{Undersampled Images}
\begin{itemize}
\item Can use PSF model as interpolant if we also want to convolve with PSF (as in likelihood coadds).  Otherwise impossible?
\end{itemize}
\subsubsection{Irregularly-Sampled Images}
\begin{itemize}
\item Approximate procedure for fixing small-scale distortions in pixel grid.
\end{itemize}

\subsection{Image Coaddition}
AUTHOR: Jim
\begin{itemize}
\item Can do outlier rejection (but usually doesn't).
\item Needs to propagate full uncertainty somehow.
\item May need to propagate larger-scale per-exposure masks to get right PSF model or other coadded quantities.
\item Should be capable of combining coadds from different bands and/or epoch ranges ranges as well as combining individual exposures.
\end{itemize}

\subsection{DCR-Corrected Template Generation}
AUTHOR: Simon
\begin{itemize}
\item Somwewhat like coaddition, but may need to add dimensions for wavelength or airmass, and may involve solving an inverse problem instead of just compute means.
\end{itemize}

\subsection{Image Decorrelation}
\subsubsection{Difference Image Decorrelation}
AUTHOR: Simon
\begin{itemize}
\item Fourier-space (?) deconvolution of preconvolved difference images before measurement - ZOGY as reinterpreted by Lupton (could apply correction in real space, too)
\item Need to test with small-scale research before committing to this approach.
\end{itemize}

\subsubsection{Coadd Decorrelation}
AUTHOR: Jim
\begin{itemize}
\item Fourier-space/iterative deconvolution of likelihood coadds, as in DMTN-15.
\item Need to test with small-scale research before committing to this approach.
\end{itemize}

\subsection{Star/Galaxy Classification}
AUTHOR: Jim
\subsubsection{Single Visit S/G, Pre-PSF}
\begin{itemize}
\item Select stars to be used in PSF estimation (mostly from moments).
\end{itemize}
\subsubsection{Single Visit S/G, Post-PSF}
\begin{itemize}
\item Extendedness or trace radius difference that classifies sources based on single frame measurements that can utilize the PSF model.  Used to select single-frame calibration stars, and probably aperture correction stars.
\end{itemize}
\subsubsection{Multi-Source S/G}
\begin{itemize}
\item Aggregate of single-visit S/G post-PSF numbers in jointcal.
\end{itemize}
\subsubsection{Object Classification}
\begin{itemize}
\item Best classification derived from multifit and possibly variability.
\end{itemize}

\subsection{Variability Characterization}

Following the \DPDD{}, lightcurve variability is characterized by providing a series of numeric summary `features' derived from the lightcurve. The DPDD baselines an approach based on Richards et al. \cite{Richards11}, with the caveat that ongoing work in time domain astronomy may change the definition, but not the number or type, of features being provided.

Richards et al. define two classes of features: those designed to characterize variability which is periodic, and those for which the period, if any, is not important. We address both below.

All of these metrics are calculated for both Objects (\DPDD{} table 4, \texttt{lcPeriodic} and \texttt{lcNonPeriodic}) and DIAObjects (\DPDD{} table 2, \texttt{lcPeriodic} and \texttt{lcNonPeriodic}). They are calculated and recorded separately in each band. Calculations for Objects are performed based on forced point source model fits (\DPDD{} table 5, \texttt{psFlux}).  Calculations for DIAObjects are performed based on point source model fits to DIASources (\DPDD{} table 1, \texttt{psFlux}). In each case, calculation requires the fluxes and errors for all of the sources in the lightcurve to be available in memory simultaneously.

\subsubsection{Characterization of periodic variability}

\begin{itemize}

\item{Characterize lightcurve as the sum of a linear term plus sinusoids at three fundamental frequencies plus four harmonics:
\begin{align}
y(t) &= ct + \sum_{i=1}^{3} \sum_{j=1}^{4} y_i(t|j f_i) \\
y_i(t|j f_i) &= a_{i,j} \sin(2 \pi j f_i t) + b_{i, j} \cos(2 \pi j f_i t) + b_{i, j, 0}
\end{align}
where $i$ sums over fundamentals and $j$ over harmonics.
}
\item{Use iterative application of the generalized Lomb-Scargle periodogram, as described in \cite{Richards11}, to establish the fundamental frequencies, $f_1$, $f_2$, $f_3$:
\begin{itemize}
  \item{Search a configurable (minimum, maximum, step) linear frequency grid with the periodogram, applying a $\log f/f_N$ penalty for frequencies above $f_N = 0.5 \langle 1 / \Delta T \rangle$, identifying the frequency $f_1$ with highest power;}
  \item{Fit and subtract that frequency and its harmonics from the lightcurve;}
  \item{Repeat the periodogram search to identify $f_2$ and $f_3$.}
\end{itemize}
}
\item{We report a total of 32 floats:
  \begin{itemize}
  \item{The linear coefficient, $c$ (1 float)}
  \item{The values of $f_1$, $f_2$, $f_3$. (3 floats)}
  \item{The amplitude, $\mathrm{A}_{i, j} = \sqrt{a_{i, j}^2 + b_{i, j}^2}$, for each $i, j$ pair. (12 floats)}
  \item{The phase, $\mathrm{PH}_{i, j} = \arctan(b_{i, j}, a_{i, j}) - \frac{j f_i}{f_1} \arctan(b_{1,1}, a_{1,1})$, for each $i, j$ pair, setting $\mathrm{PH}_{1, 1} = 0$. (12 floats)}
  \item{The significance of $f_1$ vs. the null hypothesis of white noise with no periodic signal. (1 float)}
  \item{The ratio of the significance of each of $f_2$ and $f_3$ to the significance of $f_1$. (2 floats)}
  \item{The ratio of the variance of the lightcurve before subtraction of the $f_1$ component to its variance after subtraction. (1 float)}
  \end{itemize}
NB the \DPDD{} baselines providing 32 floats, but, since $\mathrm{PH}_{1,1}$ is 0 by construction, in practice only 31 need to be stored.
}
\end{itemize}

\subsubsection{Characterization of aperiodic variability}

In addition to the periodic variability described above, we follow \cite{Richards11} in providing a series of statistics computed from the lightcurve which do not assume peridoicity. They define 20 floating point quantities in four groups which we describe here, again with the caveat that future revisions to the \DPDD{} may require changes to this list.

Basic quantities:

\begin{itemize}
\item{The maximum value of delta-magnitude over delta-time between successive points in the lightcurve.}
\item{The difference between the maximum and minimum magnitudes.}
\item{The median absolute deviation.}
\item{The fraction of measurements falling within $1/10$ amplitudes of the median.}
\item{The ``slope trend'': the fraction of increasing minus the fraction of decreasing delta-magnitude values between successive pairs of the last 30 points in the lightcurve.}
\end{itemize}

Moment calculations:

\begin{itemize}
\item{Skewness.}
\item{Small sample kurtosis, i.e.
\begin{align}
\mathrm{Kurtosis} &= \frac{n(n+1)}{(n-1)(n-2)(n-3)} \sum_{i=1}^{n} \left(\frac{x_i - \overline{x}}{s}\right)^4 -\frac{3(n-1)^2}{(n-2)(n-3)} \\
s &= \sqrt{\frac{1}{n-1} \sum_{i=1}^{n}(x_i - \overline{x})^2}
\end{align}
}
\item{Standard deviation.}
\item{The fraction of magnitudes which lie more than one standard deviation from the weighted mean.}
\item{Welch-Stetson variability index $J$ \cite{Stetson96}, defined as
\[
J = \frac{\sum_{k} \mathrm{sgn}(P_k) \sqrt{|P_k|}}{K},
\]
where the sum runs over all $K$ pairs of observations of the object, where $\mathrm{sgn}$ returns the sign of its argument, and where
\begin{align}
P_k &= \delta_i \delta_j \\
\delta_i &= \sqrt{\frac{n}{n-1}}\frac{\nu_i - \overline{\nu}}{\sigma_{\nu}},
\end{align}
where $n$ is the number of observations of the object, and $\nu_i$ its flux in observation $i$. Following the procedure described in Stetson \cite{Stetson96}, the mean is not the simple weighted algebraic mean, but is rather reweighted to account for outliers.}
\item{Welch-Stetson variability index $K$ \cite{Stetson96}, defined as
\[
K = \frac{1/n \sum_{i=1}{N}|\delta_i|}{\sqrt{1/n \sum_{i=1}{N}|\delta_i^2|}},
\]
where $N$ is the total number of observations of the object and $\delta_i$ is defined as above.}
\end{itemize}

Percentiles. Taking, for example, $F_{5,95}$ to be the difference between the $95\%$ and $5\%$ flux values, we report:

\begin{itemize}
\item{All of $F_{40,60} / F_{5,95}$, $F_{32.5,67.5} / F_{5,95}$, $F_{25,75} / F_{5,95}$, $F_{17.5,82.5} / F_{5,95}$, $F_{10,90} / F_{5,95}$}
\item{The largest absolute departure from the median flux, divided by the
median.}
\item{The radio of $F_{5,95}$ to the median.}
\end{itemize}

QSO similarity metrics, as defined by Butler \& Bloom \cite{Butler11}:

\begin{itemize}
\item{$\chi_{\mathrm{QSO}}^2 / \nu$.}
\item{$\chi_{\mathrm{False}}^2 / \nu$.}
\end{itemize}

\subsubsection{Using forced photometry}
\subsubsection{Using DIASources}

\subsection{Proper Motion and Parallax from DIASources}
AUTHOR: Simon

\subsection{Association and Matching}
\subsubsection{Single CCD to Reference Catalog, Semi-Blind}
AUTHOR: Simon
\begin{itemize}
\item Want to match in image coordinates, so also needs to transform reference catalog.
\item Run prior to single-visit WCS fitting, with only telescope's best guess as a starting WCS.
\item Single CCD form needed by AP.
\end{itemize}

\subsubsection{Single Visit to Reference Catalog, Semi-Blind}
AUTHOR: Simon
\begin{itemize}
\item Want to match in focal plane coordinates, so also needs to transform reference catalog.
\item Run prior to single-visit WCS fitting, with only telescope's best guess as a starting WCS.
\end{itemize}

\subsubsection{Multiple Visits to Reference Catalog}
AUTHOR: Jim
\begin{itemize}
\item Match sources from multiple visits to a single reference catalog, assuming good WCSs.
\end{itemize}

\subsubsection{DIAObject Generation}
AUTHOR: Simon
\begin{itemize}
\item Match all DIASources to predicted Solar System object positions and existing DIAObjects and generate new DIAObjects.  Definitely run in AP, maybe run in DRP.
\end{itemize}

\subsubsection{Object Generation}
AUTHOR: Jim
\begin{itemize}
\item Match coadd detections from different bands/SEDs/epoch-ranges, merging Footprints and associating peaks.
\item Also merge in DIASources or (if already self-associated) DIAObjects.
\end{itemize}

\subsubsection{Cross-Patch Merging}
AUTHOR: Jim
\begin{itemize}
\item Resolve duplicates in patch overlap regions by flagging ``primary'' objects.  Difficult due to blending.
\end{itemize}

\subsubsection{Cross-Tract Merging}
AUTHOR: Jim
\begin{itemize}
\item Resolve duplicates in tract overlap regions by flagging ``primary'' objects.  Difficult due to blending.
\end{itemize}

\subsection{Ephemeris Calculation}
AUTHOR: Simon
\begin{itemize}
\item Calculate positions for all solar system objects in a region at a given time.
\end{itemize}

\subsection{Make Tracklets}
AUTHOR: Simon
\begin{itemize}
\item Make all tracklet pairs
\item Merge multiple chained observation into single longer tracklets
\item Purge any tracklets inconsistent with the merged tracklets
\end{itemize}

\subsection{Attribution and precovery}
AUTHOR: Simon
\begin{itemize}
\item Predict locations of known Solar System objects
\item Match tracklet observation to predicted ephimerides taking into account velocity
\item Update SSObjects
\item Possibly iterate
\end{itemize}

\subsection{Orbit Fitting}
AUTHOR: Simon
\begin{itemize}
\item Merge unassociated tracklets into tracks.
\item Fit orbits to all tracks.
\item Purge unphysical tracks.
\item Update SSObjects
\item Possibly iterate
\end{itemize}
