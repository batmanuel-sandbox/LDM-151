\section{Preface}

The purpose of this document is to describe the design of pipelines belonging to the Applications Layer of the Large Synoptic Survey Telescope (LSST) Data Management system. These include most of the core astronomical data processing software that LSST employs.
\\

The intended audience of this document are LSST software architects and developers. It presents the baseline architecture and algorithmic selections for core DM pipelines. The document assumes the reader/developer has the required knowledge of astronomical image processing algorithms and solid understanding of the state of the art of the field, understanding of the LSST Project goals and concepts, and has read the LSST Science Requirements (\SRD) as well as the LSST Data Products Definition Document (\DPDD).
\\

This document should be read in conjunction with the LSST DM Applications Use Case Model (\appsUMLusecase). They are intended to be complementary, with the Use Case model capturing the detailed (inter)connections between individual pipeline components, and this document capturing the overall goals, pipeline architecture, and algorithmic choices.
\\

Though under strict change control\footnote{LSST Docushare handle for this document is {\tt LDM-151}.}, this is a {\bf \em living document}. Firstly, as a consequence of the ``rolling wave" LSST software development model, the designs presented in this document will be refined and made more detailed as particular pipeline functionality is about to be implemented. Secondly, the LSST will undergo a period of construction and commissioning lasting no less than seven years, followed by a decade of survey operations. To ensure their continued scientific adequacy, the overall designs and plans for LSST data processing pipelines will be periodically reviewed and updated.
